\chapter{Bamboo Poles}

As you know, I am Mwefu the king of Zinodwo. My reign began 12 years ago when I was 27 years old. I am now 39 years old and for the past ten years (under my rule) my country had destroyed and robbed this country of Kaaji, killing whoever stood in our way.
We knew our conquest was wrong, but we loved the spoils thereof too much.

One day, in the 6th year of our conquest,
\footnote{In the original the exact date is there, but Pacoitz never bothered to learn to read the Fobwa calendar.}
a group of 20 of my soldiers robbed a fruit merchant. They brought the merchant before me in my tent.

``Tell me,'' said the boy, who was also the merchant, ``is it right to rob a citizen of Zinodwo? If it is, then why not do it in your own country?''

``You are from Zinodwo?'' we asked.

``Yes.''

``What are you doing here and why haven't you returned?''

The boy muttered something; he wasn't nearly as articulate when he didn't know what he wanted to say. \footnote{The original goes on to talk about how he had planned what he was going to say on the way over, and like with all times when you plan what you're going to say, the other person says something unexpected.}

``Well?'' I interrupted.
Fobwa
``I was adrift in the flood five years\footnote{five years and ten months actually.} ago and I tried to get back. But your guards wouldn't let me back into Zinodwo because I didn't have my papers.'' responded the boy.

``So you'd like it if I had my soldiers escort you across the border?''

The merchant looked at the ground and fidgeted nervously.

``Well, I have a livelihood here, and I don't know if my family survived the flood. No offense, but I like this country very much and would like to know I have friends to go back to in Zinodwo before I leave.'' said the boy.\footnote{He was hardly a boy since according to the original, he was 15.}

``If you give me a list I'll have some of my soldiers check for your friends and relatives. What is your name?''

``Paavo.'' said the merchant as he started writing down his friends and family's names.

``Paavo, with a name like that you must be pretty timid.\footnote{\emph{Paavo} is Fo\-bwa for ``weak heart'' which figuratively means \emph{lazy}, not \emph{timid} as Pacoitz's bad translation skills might imply.}''

``I suppose. May I have my money and fruit back now?''

``Well, perhaps, but it has been three months\footnote{Pacoitz actually had this number translated properly, I suppose even he gets lucky.} since I've spoken to a Zinodwan who wasn't my soldier. I'd like to see you again. So I'll give you back your stuff and more if you come back tomorrow.''

``Thanks. But I just want what was taken, I don't need \emph{more}.''

I felt guilty for stealing Paavo's stuff, never had I felt so bad about what before I had merely considered acquisition of wealth. But now I had robbed one of my own people. So I gave Paavo back his possessions and sent him home. It wasn't just I who benefited when we raided Kaaji though, it was my whole country, for every bit of treasure I brought back, I only kept 1\% of it of myself, the rest got evenly distributed across the country; this made me very popular, and also made me wonder if I'd still be if we ever stopped plundering Kaaji.

Before the day had passed, my men and I had robbed three more merchants and 21 odd homes (They were odd because their number was not divisible by two\footnote{Oh no. Here's another terrible pun that Pacoitz added.}).
It was the first of the month, so we loaded up our horse drawn carts with all of this month's spoils and headed towards the river where we'd meet one of my ships. From there we would load up the ship and the ship would take the loot and distribute it across Zinodwo, making all my people richer.
But as we were traveling the path that went through the northern woods, the figure of a man appeared. He was clad head to toe in corroded copper; the blue-green armor was not in the least bit threatening; that is till after he ran towards us.

My guards shot arrows at him, but it was of no avail, the arrows glanced off without slowing him down. In and out of the trees he ran, very quickly for someone in what must have been a very very heavy\footnote{The original says 110lbs.} suit. When he reached the carts, he pulled the pins off the axles on one side of each of the carts. This of course caused the wheels to fall off, tipping the carts over and spreading the jewels and coins all over the mossy ground. My soldiers tried to kill him with swords, but the copper armor was stronger than steel and the blades could not cut or dent it. 

Not wielding a sword of his own, the copper man caught a swinging blade in his armored hands. He twisted the sword out of the soldier's hand and struck him with the hilt before throwing the sword into a nearby pond where it sank to the bottom. He proceeded to do likewise with six others. So, seeing swords were of no use, three of my soldiers tackled the copper clad soldier. He could not win in a simple wrestling match, so he squirmed out from under their grasp and ran back into the deeper part of the woods.

``That,'' said Twizwa,\footnote{Fo\-bwa for ``heavy servant''. Names in Fo\-bwa are given to you when you're born, (though sometimes they are changed, as it was with Mwefu) so it's most likely that Twizwa was a fat baby.} my head soldier, trembling, ``was a man we killed last year. He ran our ship aground, so we tied him up and burned him at the stake while he was yet in his armor. Then we removed his ashes and cast the suit into the river.''

My heart sank at these words that Twizwa said, but I didn't want to admit this.

``So what if he's a ghost? If we killed him once, we can kill him again.'' said I.

We had lost a great deal of time, and the journey would not be able to be completed till the next day, so my men stopped for the night.
We picked up the treasure and left it in the carts (which we kept the wheels off of to prevent the pins from being pulled out and toppling the cart) and I left 30 men to guard it for the night.

I, surrounded by the rest of my guards, went back along the path to our camp and I retired to my tent and slept, but not too well.

\tbreak

When the sun rose, I got up and remembered Paavo and the note he had written me. I gave word to one of my messengers to go aboard my treasure ship before it left that day to Zinodwo and go inquire about the list provided by Paavo. 

Well it was time for breakfast, so I ate like a king. I had eggs, rhubarb crisp, roasted almonds, and toast with apricot preserves. \footnote{Yes the ``ate like a king'' is not in the original, Pacoitz is pitifully trying to be funny again. And not even the most learned Fo\-bwa scholars can figure out what the king ate for breakfast; but they are fairly certain that he didn't have any of the things that Pacoitz claimed he did.}
I was glad that I wasn't like the poor people of Kaaji who could only afford disgusting food such as escargot and grass-fed beef. \footnote{Pacoitz is going for irony here, but none of this is in the original! Pacoitz does not belong translating great works if he is going to behave as if he were the writer!}

I waited for Paavo to show up; and waited.
It was half past noon when I finally considered that perhaps he was not coming.
Why should I have regretted taking his possessions if he was just a lazy
\footnote{See \emph{footnote 5}.} greedy merchant with no respect for the crown?!
Truly if anyone deserved my wrath it was --- and my doorman appeared.

``Paavo is here to see you,'' he said, ``shall I let him in?'' My anger subsided.

``Why yes. Take him that painting of me.\footnote{Another superfluous joke.} Just kidding, let him in.'' Paavo entered, eating an apricot\footnote{According to his biography, Pacoitz was very fond of apricots and tried to find ways to put them in everything he did.}.

``Would you like some?'' he said, wiping the juice off his chin.

``No thank you. It is good for you to be in my presence.'' We said.

``What do you mean by that?'' asked Paavo.

``Oh, it's just kingly speech. It means `I'm glad to see you.' ''

``Why do it at all?''

``Because we are the king ---''

``You and \emph{I} are the king?''\footnote{Pacoitz and his English-based foolishness again.}

``No, I alone am king, and the king only uses superior (often confusing) speech because he is always right.'' Well, at least I wanted to believe that I was always right.

``What about people that disagree with you, you the king?'' said Paavo.

``They die of course.''

``Sounds harsh.''

``It's necessary so that the king can remain in power. By the way, I sent my messenger with the list of names you gave me.''

``Thanks, I appreciate it.''

``Have you ever fought before, Paavo?''

``No, not really.''

I picked up two bamboo poles and tossed one to Paavo. I told him to fight me, but he could not wield the rod very well. I knocked it out of his hands many many times. Whenever I went to strike him, he caught my pole with his hands instead of deflecting it with his pole.

``Hold on,'' I said, ``if these were real swords instead of bamboo poles you would not be able to do that, your hands would be chopped off. You can't just catch swords in your hands.''

``I know,'' replied Paavo, ``it's just the first thing I think of when something is flying at me.''

We tried again, and again; we fought all day into the night. He was getting better, but he was still bad. At least now he wasn't catching the bamboo in his bare hands. I won every single time (That doesn't really reflect much on his inability though since I'm very good with swords).

``Are you coming back tomorrow?'' I
\footnote{I knew one high school English teacher who read this version of the book, and she saw king Mwefu's lack of using the royal ``we'' around Paavo as proof that Mwefu feels more like a normal person (and less like a king) around Paavo.
But there's absolutely no evidence in the original that the King feels more normal around Paavo at this point. 
Even Pacoitz himself admitted that he only translated it that way because it was too confusing when the king used ``we'' in situations where it could be confused to be something other than the royal ``we''. To mistake Pacoitz for a great translator is to deny the truth of his destruction of real literature.
} asked.

``I don't know.'' said Paavo, ``I'll probably be too bruised and sore in the morning to even get up out of bed.''

So Paavo went home (wherever that is). My soldiers arrived back with their spoils. I went to bed.


