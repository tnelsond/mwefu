\chapter{Regret}

I made my way back to Pwiibaa in Kaaji.

``It's simply dreadful.'' Said Twizwa, ``That a boy who you thought your friend would do such a thing to you.''

``Well, let us move on to the next village and see if perhaps we'll not encounter any more opposition.''

Several weeks passed,\footnote{two months} and we amassed much treasure, but did not encounter another rebel or ``ghost.'' Paavo was dead.

I remembered back to the degenerate people of my country -- beggars. I could have told Twizwa to stop the raid, since it obviously was doing no one any good; but would the people have sided with me? No! Of course not. I'd have been replaced with a tyrant and they'd be even worse off for it. If I were to put an end to it now, then Paavo died for nothing; he died because I would not relinquish my throne. However, if I were to spoil my own conquest, without anyone knowing, I could have both my throne and my conscience. I could honor Paavo's death by taking up his mantle (or his armor rather) and fighting my soldiers when they go to plunder Kaaji.

So I did just that and fought my own men in secret to stall the raids so that Kaaji and Zinodwo would be less full of dependent beggars.

\tbreak

``I thought you would have the sense at least,'' Said Twizwa disapprovingly as he interrupted my story, ``to only tell a story before your own execution if it would actually vindicate you. But it is clear not that you have betrayed us on nothing more than the word of the man who conspired to kill you five years ago.''

``You masqueraded as a hero,'' Said a Kaaji woman, ``when all this time you had the power to stop \emph{your own men!} You were a king and you did not lift a finger to help us while you were pampered on your throne!''

``Plainly,'' Said Twizwa addressing the woman, ``he has wronged both of our countries; played us both. Let him be hanged!''

I had nothing to say -- I had already said all they had care to hear, and what had seemed justified before, was now just cowardice on my part. In trying to keep my crown, I had become a burden to both countries and myself and Paavo --the chief offender on four sides! I should have tried diplomacy; No. I should have chosen a cause -- one cause -- and devoted myself to it wholly. I should have denied my security as king and defended Kaaji. In trying to satisfy everyone, I had ruined everyone (and Paavo).

Others always rose up to oppose me;\footnote{There goes Pacoitz, inserting his ``philosophical intro'' into the story once again.}
I wouldn't mind so much if it was just them.
For when I count my enemies and find myself in their ranks, I realize what a wretched creature I am.
It's possible (though highly improbable) that I could overcome everyone else, but how could I possibly overcome myself since I am my own equal?

What a wretched creature I am. They pulled the noose tight on me.

``Aren't you going to remove me from these chains, it's my final request.'' I said.

``Death first.'' Said Twizwa softly.

Before telling my story I had planned on escaping my fate; but after Twizwa's speech I knew that it was better -- nay my right to die. I had betrayed everyone.

``Pshaw,'' said Twizwa changing his mind, ``remove his chains. Even if he were to (which he can't) escape, he'd be apprehended -- very shortly. Not a single person here would help him.

So my chains were taken off and the trapdoor on which I stood released. I tried to untie the noose as it choked my neck; it was supposed to be escapable, I had tied the crummy knot myself. My escape plan useless, I was filled with relief at not having to live any longer -- to commit any more terrible selfish mistakes.

But just as I was about to die, a man named Pakoja shot an arrow and it cut me from the noose. Then in an explosion of light -- that powerful angelic being who is the paragon of intelligence, strength, and morality -- knocked all who were there at my execution to the ground. Then he picked me up and alighted me to the next town. Then Pakoja was gone.\footnote{If you could not tell by the sheer ridiculousness of this Pakoja character, it is a Pacoitz insertion. In the original, Mwefu did manage to untie the knot and escape on horseback. Pakoja is clearly just an idealized version of Pacoitz himself. How conceited! He obviously perceives himself as perfect and wholly worthy of being in the story; both of these claims are laughably arrogant and ignorant.}

I fled to the Woolly Wood in the eastern mountain range of Kaaji. I hunted my own food, made my own shelters, and avoided contact with people. Several days passed.\footnote{27 days}

\tbreak

Then a man approached me; his armor was well-crafted and he pointed the tip of a longsword towards me.

``You! Help me.'' He said, ``I'm looking for Zinodwo's ex-king Mwefu.''

``I'm in the middle of forest in Kaaji, why would I have seen the king of Zinodwo here? And why do you say ex-king? What do you mean by that?'' I replied.

``Only that he betrayed his country and did not help your country at all, and now there's a price on his head.''

``Are Zinodwo and Kaaji working together!?''

``Yes, they've made a pact and the raids and fighting have stopped until the traitor is at last dead.'' Naazwato couldn't imagine how comforted I was to learn that. Here was a reason for living, because what was not possible for me when I was king came to be by my becoming an outlaw. 

``Too bad that their peace is only to kill me.'' I thought to myself, ``As well I deserve. But if I die it will all be over; so I have to give them a chase.''

``You're a bounty hunter then?'' I asked.

``Indeed, and my armor bearer deserted me in an effort to get all the bounty for himself.'' 

I couldn't let him know about my expertise in combat, lest he realize who I was.

``Perhaps,'' I thought to myself, ``he suspects me of being Mwefu. If I were to act as a servant and not as a king he might not suspect me. (Who would suspect an armor bearer of being the king?)''

``If it pleases my lord,'' I said, ``I will bear your armor and help you capture Mwefu; though I must confess that I am not a good fighter.''

``Just carry my armor. You'll get five percent of the bounty.''

``It's agreed, then. My name is Goat.''

``And I am Naazwato, future king of Zinodwo.''

So I feigned ignorance and journeyed with Naazwato as his servant -- me who had been king, brought so low. We spent the next four days exploring the Woolly wood but found nothing that could be linked to Mwefu (or so I let him believe). Finally we descended back down to the base of the mountain and stopped at an inn.

``Did you find that treacherous abomination Mwefu?'' Said the innkeeper.

``No.'' Said Naazwato, ``All I found in the mountains was this wretch who I've brought on as my armor bearer.''

``That's fortunate,'' commented the innkeeper's daughter, ``since you killed your last one I would have thought you'd have trouble getting another.'' At this accusation my ``master'' turned pale.

``No I did not.'' He retorted.

``Yes you did, the grave's out back.'' She said cooly.

``No use arguing with children. You know what they say, kids are as stubborn as goats; because well, they are goats.\footnote{This was in the original language, (more or less) but it was not a pun on the meaning of \emph{kid}. Pacoitz apparently left it as is because his base interests found it ``amusing.''}

``I am faithful.'' I lied to Naazwato, ``While I'm not always loyal, I fear the one who has the power to take my life.''

``As will everyone come my time.'' Replied Naazwato.

We stayed the night at the inn. My ``master'' and I took our turns telling stories -- I tried to tell ones which seemed more Kaaji and less Zinodwo. To write down the stories we each told would be unnecessary, but here is a rather good one which Naazwato told:

``When Mwefu the scum was young, he aspired to become a fighter, but due to his poverty he could not acquire neither proper weapons nor teachers. But he drove himself to learn. He watched the great fighters who were the rebels and rioters of Zinodwo, and over time he defeated them.''

``I've heard as much, but how ever did he manage that?'' I asked feigning ignorance and curious to know what this man thought of me.

``He took them on one at a time over the course of ten years. Mwefu noticed that each fighter usually had one trait, one skill, which was so developed as to make him unbeatable. So Mwefu took his time mastering the technique of the rebel till he was indeed better at it than the rebel himself who he was copying.
Then Mwefu would challenge the rebel to a fight and humiliate him to death. In this way, he was able to rid Zinodwo of undesirable thieves and troublemakers while becoming the best warrior in the known world. Soon after, he lusted for the throne and demanded that a tournament for the crown be arranged (which is how a king is chosen in Zinodwo). Needless to say, none was his equal and he became king. But who could've known that he would be the most selfish and treacherous king ever to ascend the throne?''

``That is a very interesting story.'' I said, ``But in light of that, how do you ever think we will be able to overcome this traitor once we find him?''

``Oh that's simple, because I am better.''

``Indeed?''

``Indeed.''

\tbreak

The next morning, Naazwato paid his dues to the innkeeper and we left the Woolly wood and crossed a papaya field. I said something which doubted his ability; then he swung his leg at me to knock me down; but I couldn't control my reflexes, and before I knew it, I had counter-kicked his leg and sent him tumbling backwards.

``You said you couldn't fight.'' He said with a shudder. I feared that he had figured out that I was really the ex-king.

``But I can't fight.'' I lied, ``I am however a wiseman who could probably devise strategies for fighting -- but my body is not fast, nor strong, nor trained like yours. That move I performed was just something I came up with while inspired by your Mwefu story.''

So we fought again and I tried very hard to let him beat me; and so he did.

``Upstart.'' He said, ``Do you still doubt me. Just because you got on lucky hit on me does not mean that I can't take on Mwefu.''

``No. You have proved yourself. But rather than wander Kaaji and not find Mwefu, wouldn't you be better off becoming king of Zinodwo?''

``No although I aspire to be king, (because I deserve it) I missed my opportunity because I was on a tradeship across the sea when the tournament for the crown was announced. And as it is I do not yet have enough standing to request another tournament; But once I catch Mwefu I will.''

So we continued on and I resolved to not let myself be found out -- not for my sake, but for that of Zinodwo and Kaaji whose peace would only last while they both hunted me.
