\chapter{The Monster}
Naazwato took me deeper into the heart of Kaaji till\footnote{He means 'til.} we came to the Desecrated Valley.

``Why in Mwefu's name would you bring us to this accursed place?'' I asked actually afraid.

``Are you really that n{\"a}if?\footnote{na{\"i}f}'' Said Naazwato, ``One monster is just as good as another. If I can't kill Mwefu I'll kill this so called devourer.''

``The difference -- is that you can actually kill Mwefu.''

``Don't tell me you believe all those myths and wives tales.''

``Why would you try to kill something you don't believe exists then?''

``Oh it likely exists, but greatly exaggerated; maybe it's just a badger or lion.''

We soon came upon a massive cage that had three iron barred walls and the fourth was a cliff face. But there was no monster to be seen; the bars, which could hardly be called that anymore, for they were bent every-which-way and had impressions of massive hands.

Then we heard a cry, a sad and passionate wail.

``Let us leave before we too are the ones making that sound.'' I pleaded.

``Do you think I'm a coward!? Come let's seek out that sound.''

We climbed through a thin forest up the valley to a graveyard where the most frightful creature was cutting himself on the tombstones like a dog with fleas.
He was the figure of a man, but a good two times taller and covered in fur.
It was a sunny day, but a shadow fell on the beast and moved as he moved, not even the sun, who has shined on every being since creation would waste its warmth on this foul thing.

I was very afraid and could see that Naazwato was too. The creature galloped toward us and we drew our swords and slashed at it. But though it bled more, that was not enough to stop it; the creature swept me up with one arm, and Naazwato in the other. We all three screamed in agony and it thrashed us back in forth.
Then it dropped us and said, ``Suffer me to live!'' But he wasn't talking to I\footnote{grammar} or Naazwato, but to a feeble-looking unarmed traveler.

``I'll suffer you to die. Release him!'' Said the man; and a white flaming figure seemed to leap out of him and grabbed the beast before renting it in twain\footnote{two}.
And where the creature had been there was a naked man. Then the traveler threw his cloak over the man and embraced him.

``You've saved me.'' Said the former beast overflowing with gratitude. 

``He's not defeated yet. I have to become you, so you can become me.'' Said the traveler as a shadow passed over him. Then the traveler let out a scream more terrifying and filled with more pain then the one of the beast, as he turned into one. The former beast ran away from his rescuer-now-beast\footnote{That's not even an idiom.}.

``Maybe we can kill this one.'' Said Naazwato.

``But he's not a monster, he saved that man by taking his place--''

``He may have not been before, but he is now, someone ought to put him out of his misery before he slaughters anybody.''
The creature made no attacks but writhed in agony. Naazwato began to strike it with his sword, and I did also. But still the monster did not return a blow.

Until at last, every drop of its blood was spilled, and the sky went suddenly dark. His cries had finally stopped.

``Wrap the body, we're taking it to Zinodwo.'' Commanded Naazwato.
When the light returned, I wrapped the body, which was now that of a man, and placed it on a beast of burden.

``This isn't much of a trophy anymore is it? We killed an innocent man and have nothing but a tortured body to show for it.''

``Shut up.''

After several days of walking north toward Zinodwo we had lost the body.

``It couldn't have just walked away by itself.'' I said.

``Of course not, that's because you up and buried it.'' Said Naazwato.

``I did not, but anyway the body would have been all rotted by the time we got back anyway.''

``Well then we'll get the former beast, he's the guilty one.''
After a week of searching we found him -- he was inside his old cage weeping.

``You should have killed me instead, that was my sickness, it was supposed to be me!'' He said.

``Il soupira profond{\'e}ment. Elle cependant lui souriait avec ce sublime sourire auquel il manquait deux dents.'' I said. \footnote{Translates to: ``He sighed deeply. However, she smiled with that sublime smile that was missing two teeth.'' As you can tell that is neither in the original, nor does it have any bearing on the story. Pacoitz is just trying to make the reader believe that he is sophisticated and can write French when it's clear to anybody that all he can really do is open up Victor Hugo and copy the first sentence he reads. I once read a work that was so interspread with French sentences that I found myself reading the footnotes rather than the actual book, which I suppose is what you the reader have been rightly doing this whole time. What is really disheartening is when those sentences add nothing to the story and one realizes that they can be skipped with their footnotes entirely and one will not miss out because the sentences were only there to pad the book -- and the ego of the author.}

``Well then, let's kill the beast and go claim my kingdom.'' Naazwato said.

Having compassion on the former beast, I spoke up, ``Naazwato, he's not the beast anymore, couldn't we just leave him be?''

``Not unless I want to ignore my destiny and let the world collapse.'' Said Naazwato.

``What if we found Mwefu?''

``Well in light of that we could.''

``I have a confession then, I am he.''

Naazwato let out a snicker and said, ``That's quite an amusing claim, but the resemblance is there I must admit. Whether you are or not makes no difference to me provided you can convince everyone else that you are.''

``That I probably could. Though we'll have to make some preparations.''

I led Naazwato to Jaahwii and was greeted by the crippled nuisance.

``Ahoy, you've made it alive this far Mwefey my boy.'' Said the beggar.

``So he really is Mwefu?!'' Said Naazwato exasperated.

``You take the word of a beggar as fact?'' I asked.

``Well you're clever enough to fool him, but not me. What's important is that everyone else believes it.'' Naazwato retorted.

The beggar and I conversed and I paid him for one of his failed poems, for a reason that will later be clear.

Naazwato then bound me up, and the two of us boarded a merchant vessel headed for Zinodwo.
When we arrived, we hired a wagon to take us to the capital. It took three days and then we waited for an audience with the king.

``Who is this new king?'' I asked.

``Some bandit from the north.'' Said Naazwato, ``This system for determining the fate of the country on the outcome of a silly tournament is nonsense.''

The king's guards led us in to his throne room. It was filled with gold and jewels, the obvious overflow from what was now an overfull treasure room.

``So this is the traitorous snake Mwefu?'' Said the king.

``Indeed.'' Said Twizwa who sat on the king's side, ``Naazwato, you would like the three and a quarter pound\footnote{Ten vwas/zwas; but no one but Pacoitz has the audacity to actually claim an imperial weight off some unknown unit.} of gold reward then?''

``Actually,'' I said standing up straight, ``Naazwato feels he's entitled to start an attempt at becoming king.''

``Silence!'' Said Twizwa, ``What kind of fools do you take us for?''

``Ones with the honor to see that your last tournament was lacking a real talent.'' I replied.

``He'll lose just as everyone else lost.'' Said the king.

``I apprehended the biggest threat to the crown, I deserve a chance.'' Said Naazwato.

``Very well,'' Said the king, ``guards, have Mwefu beheaded and prepare the country for the tournament of kings.''

``Wait!'' Said Naazwato, ``For the moment this snake is my armor bearer and I need his aid in the contest.''

``Fine!'' Said the king, ``Postpone his beheading until after the tournament two weeks hence.''

So for two weeks I trained Naazwato and he got better and better at combat, but I was not very convinced that he could win.\footnote{Pacoitz actually ends his translation (if you can even call it that) right here. The original text goes on for another 30 years of intricate plot. So to remedy this injustice you can read \emph{Appendix C} where I finish the story properly.}
