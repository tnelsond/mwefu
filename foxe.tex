\chapter{Sarah Foxe's Continuation}
Charles gave me this copy out of the trash and it's actually better than Graft's finished copy, so I'm substituting that with this.
I also found some unpublished essays by Graft and others that could really use a home in a published work.
\begin{flushright}
\textsc{
Sarah Foxe,\\
Typesetter of \emph{Whinery Press}}
\end{flushright}

\section{The Problems Inherent in Authors}
\begin{flushright}
\textsc{
by Robert Graft}
\end{flushright}

Sure nobody is perfect, but nevertheless, it is important to note their faults.
Take fiction authors for instance, many of them either use a pseudonym or an overaggressive abbreviation to mask their identities: Mark Twain, A. A. Milne, J. R. R. Tolkien, et al.
This dissociativeness from their real names shows their insecurities, namely dissatisfaction with who they are, and a compulsion to create new worlds to numb the disorder of their lives.

Now those authors of fiction who do not hide their names have another issue: They are arrogant and seeking praise for themselves.
One would then wonder if it were possible to write fiction without fault. No it is not. Whatever fault you possess will be evident by the way you sign it.
Sure you could sign the opposite way you'd like, but you won't. Because if writers of fiction had \emph{any} self-control, they would not be writing fiction.

Do not doubt for a second that non-fiction authors have their own problems. When faced with chronicling a rather boring event, their pride drives them to make it somehow sound interesting and epic through various superfluous embellishments. But in doing so, they have incidentally created an ungodly hybrid between non-fiction and fiction, having all the detractions of both and the benefits of neither.

As readers, we enjoy non-fiction because it is true. When a writer abuses this by sprinkling his work with lies and exaggerations and ``literary elements'' he betrays his readers.
It should not be required for a reader of a work of non-fiction to still have to sift the fact from the fiction.
If the author hates the truth so much, he should write straight fiction and label it as such so as not to deceive his audience.

\section{Forest for the Trees}
\begin{flushright}
\textsc{
by Robert Graft}
\end{flushright}
There is no greater enemy of the English language than the ones who've been appointed to defend it, namely English teachers. They dissect mediocre works of literature like a man without teeth puts pasta into a blender. All of the texture is thus lost, the flavor becomes a bland mix, and they wonder why their students fail to do the assignments.
Literary elements did not exist until the advent of public schooling wherein teachers decided that narratives didn't matter and all that mattered was fine insignificant details.

Literary elements are the devil's substitute for real writing talent. By that I mean that the author intentionally uses them to muddy the story because he knows that that his writing will not hold up to fine scrutiny. By utilizing literary elements, he redirects attention away from plot, characters, humor, emotion, and logic, and redirects it towards vain tangles of words.

However, with skilled writers who don't pleasure in such esoteric banality, the teachers still strain to find the ``literary elements.''

``Why did the author use this technique?'' They'll ask, when the answer is that there is no technique and they are merely seeing the floaters within their own eyes.

Teachers never select stories because they're good stories or because the writing is good, they select them based on insignificant details about the author.

``This author was a woman with dark skin color.'' Well that does not make a bad story a good one, regardless in how much one believes in social justice.

Or take ``The Crucible'' which teaches students that we should just let the communists win because they don't exist.
The communists have won, they wrote the book and they are pushing it on students to weaken their aversion to broken economic systems. Nothing else about the book is noteworthy, and it is nothing more than a subversive brainwashing tool.

In a way it's reverse subliminal messaging. Normally the true message is hidden and only picked up by the subconscious, but with reverse subliminal messaging the true message is apparent but the focus is placed on ``literary elements'' and tiny details. The same effect is achieved by both methods. It's hard for students to criticize the message of a story when they're not aware there is one because they're too busy reading it trying to find ``hidden meanings.'' Not sure whether the irony of that is lost on the teachers, because as I recall they always did love pointing out irony, even if they were incapable of explaining what it was to students.

\section{}
\begin{flushright}
\textsc{
by Robert Graft}
\end{flushright}
