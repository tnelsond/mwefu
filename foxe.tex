\chapter{Sarah Foxe's Continuation}
I also found some unpublished essays by Graft and others that could really use a home in a published work. Mainly they're just bizarre.
\begin{flushright}
\textsc{
Sarah Foxe,\\
Typesetter of \emph{Whinery Press}}
\end{flushright}

\section{The Problems Inherent in Authors}
\begin{flushright}
\textsc{
by Robert Graft}
\end{flushright}

Sure nobody is perfect, but nevertheless, it is important to note their faults.
Take fiction authors for instance, many of them either use a pseudonym or an overaggressive abbreviation to mask their identities: Mark Twain, A. A. Milne, J. R. R. Tolkien, et al.
This dissociativeness from their real names shows their insecurities, namely dissatisfaction with who they are, and a compulsion to create new worlds to numb the disorder of their lives.

Now those authors of fiction who do not hide their names have another issue: They are arrogant and seeking praise for themselves.
One would then wonder if it were possible to write fiction without fault. No it is not. Whatever fault you possess will be evident by the way you sign it.
Sure you could sign the opposite way you'd like, but you won't. Because if writers of fiction had \emph{any} self-control, they would not be writing fiction.

Do not doubt for a second that non-fiction authors have their own problems. When faced with chronicling a rather boring event, their pride drives them to make it somehow sound interesting and epic through various superfluous embellishments. But in doing so, they have incidentally created an ungodly hybrid between non-fiction and fiction, having all the detractions of both and the benefits of neither.

As readers, we enjoy non-fiction because it is true. When a writer abuses this by sprinkling his work with lies and exaggerations and ``literary elements'' he betrays his readers.
It should not be required for a reader of a work of non-fiction to still have to sift the fact from the fiction.
If the author hates the truth so much, he should write straight fiction and label it as such so as not to deceive his audience.

\section{Forest for the Trees}
\begin{flushright}
\textsc{
by Robert Graft}
\end{flushright}
There is no greater enemy of the English language than the ones who've been appointed to defend it, namely English teachers. They dissect mediocre works of literature like a man without teeth puts pasta into a blender. All of the texture is thus lost, the flavor becomes a bland mix, and they wonder why their students fail to do the assignments.
Literary elements did not exist until the advent of public schooling wherein teachers decided that narratives didn't matter and all that mattered was fine insignificant details.

Literary elements are the devil's substitute for real writing talent. By that I mean that the author intentionally uses them to muddy the story because he knows that that his writing will not hold up to fine scrutiny. By utilizing these distractions, he redirects attention away from plot, characters, humor, emotion, and logic, and redirects it towards vain tangles of words.

However, with skilled writers who don't pleasure in such esoteric banality, the teachers still strain to find the ``literary elements.''

``Why did the author use this technique?'' They'll ask, when the answer is that there is no technique and they are merely seeing the floaters within their own eyes.

Teachers never select stories because they're good stories or because the writing is good, they select them based on insignificant details about the author.

``This author was a woman with dark skin color.'' Well that does not make a bad story a good one, regardless of how much one believes in social justice.

Or take ``The Crucible'' which teaches students that we should just let the communists win because they don't exist.
The communists have won, they wrote the book and they are pushing it on students to weaken their aversion to broken economic systems. Nothing else about the book is noteworthy, and it is nothing more than a subversive brainwashing tool.

In a way it's reverse subliminal messaging. Normally the true message is hidden and only picked up by the subconscious, but with reverse subliminal messaging the true message is apparent but the focus is placed on ``literary elements'' and tiny details. The same effect is achieved by both methods. It's hard for students to criticize the message of a story when they're not aware there is one because they're too busy reading it trying to find ``hidden meanings.'' Not sure whether the irony of that is lost on the teachers, because as I recall they always did love pointing out irony, even if they were incapable of explaining what it was to students.

\section{The Nature of Medicine}
\begin{flushright}
\textsc{
by Robert Graft}
\end{flushright}

``Medicine is the pinnacle of evolution and technology and society.'' Except it's not. It was founded by Nazis who tortured and experimented on Jews. These same ``doctors'' such as Joseph Mengele and others were then forgiven and brought over to the United States to practice medicine here. I'm sure their intentions are good this time around though, right? Who better to trust with our lives than Nazis?

You feel sick? Just take a pill, why not? Does the pill even fight the symptoms of what you have? Who cares. Just take as many contradictory medications as you can, even a doctor can't predict what will happen, but go ahead, trust them to prescribe you more and more.

Oh, are you afraid of getting sick? Just inject a dead baby or a dead rabbit straight into your bloodstream, what could go wrong? Don't worry, we also added some good things to the vaccines that way it's not just dead babies, we added mercury, preservatives, aluminum, and who knows what else. But hey, just remember that you'll still get sick unless we can get 100\% of the people to get vaccinated; because that's how effective it is, they only work if absolutely everyone takes them (in theory). Pardon me if I don't trust in that. Oh but Polio is eradicated now; yes indeed, everyone that could die from it already died, the vaccine is just trying to take credit for natural selection.

That's not entirely true, for most people Polio was innocuous. And what was diagnosed as poliovirus was also a combination of pesticide and metal poisonings which have similar symptoms. But after the vaccine was introduced, the CDC narrowed the diagnosis to exclude everything but the poliovirus.

Oh, but God intends us to use our wisdom to blindly trust Nazis with our medicine; you might say. And how is that wisdom? When did man forget how to think for himself? George Washington's doctors (not even Nazis) killed him by making him bleed out, where's the wisdom in that?

Charles says that Jesus already paid for our healing when he was whipped; I'd never heard that, but I think he's right. His prayers are effective and I've seen Jesus heal what doctors had caused.

``At least doctors are better than witch doctors.'' You might say, but I don't see the difference really. They both deal with mumbo jumbo and nonsensical magic. O, that was harsh, some of the witch doctor's remedies actually work.

I used to believe that God wanted people sick to teach them something or keep them humble. But as my wife pointed out, not even paralysis could humble me, so that's nonsense. Whenever I get sick, I lose compassion for my fellow man due to my highly developed tolerances; I can endure intense ammounts of pain and disability without it bothering me. So when I see someone complain about being sick, I tell them to stop being a baby and that I've overcome much worse. If getting better takes going to a Nazi or a witchdoctor, I think then you should do that since it's not God's will for us to be sick. Though miraculous healing is better, not everyone has a man of faith to pray for them, or can be bothered to become one. So medicine is a decent compromise.
