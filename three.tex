\chapter{Music}

Nothing of importance happened the next day. Paavo did not come to visit me; likely because he was sore from fighting me all night.

The day after that, however, Paavo arrived for breakfast. He didn't care much for the wine or finer foods, instead he ate apricots. We then practiced fighting again with bamboo poles. I won as always, but he was showing promise.

``Your highness,'' said Twizwa as he entered my tent, ``the time for the raid draws near.'' Twizwa was far better at talking like a poet than anyone in my kingdom.

Some question my choice in putting Twizwa in command of our army since he's really not a very good fighter. When we first became king,
\footnote{Pacoitz omitted the part of the story where his former head soldier, Bwaa tried to kill king Mwefu and was thus put to death. In fact in the original there was a whole chapter about it, and besides developing Twizwa's character, it made Mwefu's choice to elevate Twizwa much more credible.}
I asked Twizwa when he was just my page, we asked him, ``Twizwa, you are wise, tell me, who would you put in charge of my military?'' and he replied with, ``If I were the king, I would search my kingdom for the one whose grasp of language were unparalleled, whose way with words was unmatched.''

The truth is that words are more useful for controlling than might is, and I already am the greatest fighter we've ever faced, so why would I need another great fighter if he's just going to command rather than fight anyway? I don't even see why I was made king, just because I won a tournament? Because I took down the ten best warriors in the country? That's just how the system for the election of kings in Zinodwo works. But we're glad that's how it works because the conveniences and power of being king are most delightful.
So if my second in command should be good with words, then who else can speak as Twizwa can?\footnote{See Appendix \ref{rotshift}} So I put him in charge of the military.

I sent Paavo home and we went on the raid with Twizwa. We were camped in, and still were taking resources from the village of Vwi in Kaaji. We marched through the trees to the west and came to a place where four brick houses were.

We divided into five groups; four to rummage each of the houses, and one to play some triumphant trumpet music.
The people were forced out of their homes and the beautiful music made the feat seem glorious. They were about to all be killed, but then I noticed that someone else had joined the band, a flute was echoing the theme that our musicians were playing. It got closer and closer, till the blue-green copper clad figure appeared; he was playing a rather large bamboo flute.

My trumpeters stopped, and I told my soldiers to apprehend the flute player. They shot arrows at him, which of course bounced off his armor. He did not stop playing that flute. Then using an arrowhead, he cut the ropes that held the horses to the cart and scared them off.

``He's taunting us, isn't he?'' Said Twizwa.

``Just get the nets and remind the soldiers that even though his armor makes it impossible to cut him, blunt blows should still hurt.'' I said to Twizwa.

``You're assuming he's not a ghost.''

Twizwa gave the commands that I told him. The soldiers beat upon the rebel with their swords, ruining their edges in the process; nevertheless it seemed that it was having effect, because he was knocked to the ground. They were about to tackle him when he got up and snatched one of the swinging swords. Lo, and behold, he could wield a sword now. He disarmed and wounded many of them (They weren't wearing full armor as he was because it would be too heavy to be practical). Meanwhile, the people who we had planned on killing had fled into the woods and no soldier was left to stop them because they were too busy fighting the rebel.

Then he too disappeared into the woods just as he had come. Not one of my soldiers could catch him.\footnote{Pacoitz fails to mention that the soldiers rode horses. If it had been an open plain they could have rode to catch him no problem, but with all the trees and cliffs, and paths which are too narrow for horses to go through, there was no chance; and he was far too fast to be apprehended on foot.} If I hadn't believed he was a ghost before, I believed it now, and it filled us all with fear.

A resistance then broke out and the people whose homes we had forced them out of had come back with a militia and they drove us out of there before we could claim any of their gold or silver. I really hated that rebel.

Then the most terrifying thought occurred to me; neither the rebel nor Paavo could wield a sword when I first met them, but after I taught Paavo how to fight, the rebel showed fighting skill too. They also caught weapons in their hands. Was this enough grounds to prove they were the same person, that my good friend had betrayed his own country and was fighting for the enemy? No, no, it couldn't be, Paavo was far too frail to be such a warrior.

\tbreak

I awoke the next morning to Twizwa giving a speech about the proper way to kill an enemy or something like that.\footnote{In the original, Twizwa's speech is actually about ghosts and about how it might be possible to defeat the copper clad warrior by utilizing various supserstitions. It was an amusing and interesting speech, and it is inexcusable for Pacoitz not to put it in. To omit this part leaves out a lot of the original comedy (which was actually funny, unlike Pacoitz's puns).}
Forgetting my suspicion that Paavo might be a traitor, I taught Paavo how to wrestle; he learned very quickly and over the course of that week he got very good at it. My soldiers made some raids throughout that week without me, but each time they came back with less and less gold and more stories of the ghost of the copper clad rebel. I was sick of it, but I was too afraid to die in battle against an invincible adversary. But suppose it was Paavo? I needed to go on one last raid to find out.

So we went north towards the river and stole some silver from the people there. But the copper clad rebel ran up like usuall and my soldiers tackled him. For a moment he was on the bottom of a pile of four soldiers, but then I saw him, flip over soldier after soldier till he was at the top of it, twisting arms in a way that prevented any of them from moving. All the soldiers quaked in fear.

So now I knew that this rebel was indeed Paavo, using the same techniques I had taught him. I knew he was not a ghost, so I drew my sword. We fought, just me and him, but I was a great swordsman and I disarmed him. Then I sliced the hinges off the breastplate of his armor, leaving his chest exposed with only his shirt to protect him. Then, I stabbed him in the left of his chest. Piercing where I knew his heart to be. The music my soldiers played was triumphant, but my heart\footnote{I hate this word, it's so vague. It's definition varies to the point that it means nothing. It's clich\'{e} and Pacoitz is a fool for using it. A better word for this context would be ``emotions''.} did not listen and instead played a dirge. 

The rebel collapsed and his once quick body became as slow as the dirt. We let him lie there. I knew it was Paavo, I had no need to remove his helmet. I almost regretted killing him since he had been good company.
I figured that if anyone found out that the rebel was Paavo I'd look like a fool for befriending the enemy;
So we commanded that no one touch the body, under punishment of death. Then I realized that with the body sitting there, anybody (whether they were my soldiers or not) could take off the armor and see who was in it when I was not looking; it'd be best if I brought the dead rebel back to my camp where I could come up with an excuse to remove the body from the armor myself and keep his identity a secret.

``Let's take the body with us.'' We said.

``But you said not to touch it.'' My soldiers replied.

``Oh, yes, you're right.'' I didn't want them to think I had changed my mind about that, so I picked up the armored rebel; he was heavy, but the exertion was worth not going back on my own orders.

We then finished the raid and went back to camp where we celebrated our victory; not even the death of a good friend can stop me from enjoying good wine and admiring good gold. After that though, I had a casket made and set in one of our supply tents. Then I placed the copper rebel inside it and locked it with a padlock. But I forgot to tell someone to bury it.


