\chapter{Epiphany}

Several days passed and we were nearly finished plundering the city of Vwishaja\footnote{``Yellow child.'' However, this word is rendered improperly because the different syllables are in different cases for apparently no reason; it should be \emph{Vwifava} or \emph{Jwihaja} if we conjugate it into a verb like \emph{Kaaji} is.}. But as I was doing some shopping\footnote{More like stealing.}, I saw Paavo, selling fruit as he always did.

Paavo's fruit stand was rather disorganized, it was seven or eight planks set on three or four stumps with unsorted baskets of fruit balancing on them.

``Hey, sorry I haven't visited in awhile. I've been kinda busy selling fruit. Business has been booming.'' He said. I did not know what to make of this, I was absolutely certain that Paavo was the copper rebel, but I'd killed him, stabbed him through the heart, how could he still be alive?
Perhaps he and the rebel were two separate people after all.
Could it be that I'd misjudged him and he'd been my ally all along?
Or maybe he really is a ghost and that's how he survived -- No, no! Paavo is no ghost.

``What kind of fruit would you recommend?'' Said I to him.

``Well, there's apricots.'' He replied.

``No, I mean good fruit.''

``You don't like apricots?''

``No, not really. I'll take six bushels of oranges.'' I said as I handed him six small nuggets of gold. Paavo then took a couple minutes to separate the oranges from the rest of the fruit and managed to get only four bushels.

My men and I went on our way, and ate all the oranges we had bought in a matter of minutes. Then we went back to camp. The first thing I did was hurriedly go to the supply tent to double-check that the rebel was still inside it. But he wasn't, and the casket looked as if it had been kicked apart from the inside out. The padlock was still on it, but since there was a gaping hole in the top of the coffin, the lock had done no good. Now I know someone did not steal the body, the splinters from the hole were pointing outward, so the rebel must have come back to life. But even if the rebel had come back to life, that didn't mean it was necessarily Paavo.

I didn't know what to do, so I showed Twizwa the broken casket.

``That rebel will not stay dead.'' Said Twizwa, ``But that matters little now I suppose. From what I know, ghosts are usually local, they only haunt near the location from whence they were slain. If we were to go to the next town we might find all our ghosts problems to be gone. And besides that, we're finished with this town anyway.''

\tbreak

The next day our ship was blowing its horn in the harbor. Apparently they were ready for another shipment of spoils. I gathered most of the men and we brought our horse drawn (apparently horses are very good artists\footnote{Another clich\'{e} pun.}) carts out to meet it. My soldiers were on the look out for the copper warrior. They were very frightened at the thought of ghosts, and I was too; I was a much better fighter, and it's very unlikely that'd I lose a fight to it, but with all the coming back from the dead, sooner or later he might get lucky, and I might get a headache, and I might lose. I hoped what Twizwa said about ghosts being local was true and not merely wishful thinking. It'd be dreadful to move to the next town and have to kill it all over again, and again, and again.

As the saying goes, speak of the devil and he appears.\footnote{In the text, both Pacoitz's and the original, no one was talking about ghosts (though they all were thinking about them). And this line, besides being bad even before it became overused in everything, is not even in the original.} One of the carts tipped over and lo, the copper clad rebel had been in the cart, hiding under the money the whole time and had pulled the pins off of one side of both axles. He then leaped into the next cart and proceeded to tip it over as well.

The rebel's breastplate was still broken off and he worked hard catching swords and blocking arrows to prevent them from hitting his chest. Then when my men remembered to wrestle him, he knocked several of them down the hill that the path was going down. Then, running fairly fast, he disappeared into the woods. He only managed to tip over two of the carts, so the other one was still standing.

Then we had to fix the carts. While we were fixing them, that messenger that I had sent earlier arrived.

``I found out what you wanted me to look into about Paavo.'' The messenger said. And he went on to explain that Paavo's family was all missing as far as he could find out, but he did find some old friends of the family; they told him that when Paavo was born his parents could barely feel his heart beat, so they named him Paavo which means ``weak heart.'' No sooner had they named him though, when they realized that his heart, as it turns out, was on the other side of his chest. Paavo did not have a weak heart after all, they just checked for a heartbeat in the wrong place. But this didn't matter, because by the time his parents realized this the name had stuck.

This all made sense then. The teal colored copper clad rebel\footnote{Pacoitz constantly renders the same Fo\-bwa phrase regarding the rebel as something different each time. His unsurety is nauseating.} was Paavo, and because he was Paavo he learned as Paavo learned, and when stabbed in the left of the chest he did not die, because his heart was on the opposite side. The warrior was no ghost, and neither was Paavo.

But what was I to do? Was I to betray Paavo as he betrayed me?
No, I liked him.
I couldn't bear to bring upon him that much disgrace and to lose him to death a second time.
As long as no one found out, I would be saved from the shame (and possible revolt against me) from befriending this rebel. I would then tell no one; not even him, because then no one would ever hear me say anything to implicate myself in such a crime against Zinodwo. I was still going to make sure that Zinodwo got plenty of spoils, but I also wanted Paavo to feel like he was doing at least something noble to obstruct our conquest.

\tbreak

Because Twizwa kept nagging me about ghosts and how foolish I was being for staying in this village, we left Vwishaja and headed toward Jaa\-hwii\footnote{Fo\-bwa for ``tall and solid.''}. The city had thick walls and it had a lot of archers. But we assembled trebuchets and launched many boulders at it. We sieged it for a few hours; it didn't take long.\footnote{In the original, the city had been in ruins since the King had besieged it the year before and he had only now got around to plundering it. It is very foolish of Pacoitz to think that anyone could siege a city in just a few hours.} The wall was now in shambles and many of the buildings too. With the archers all defeated we marched into Jaa\-hwii.

On the way, a beggar stopped us, he laid right in the road. He was old and he was scruffy and he asked us for some alms.

``No,'' We said to the old man, ``we don't support beggars, it's not a good way of living.''

``You are the king of Zinodwo, aren't ya?'' Replied the beggar. Then he burst out laughing uncontrollably. It took a good minute to stop him.

``Why are you laughing? Doth a beggar have more standing than the king?'' Asked Twizwa while he violently shook the man.

``The king of the other country said he don't support beggars. Well, that's funny indeed because he created two whole countries of them. You robbed my country of Kaaji blind and made us into beggars, and then you know the rest.''

``Know the rest of what?'' I asked, ``I do everything to stop begging in my country. I even went as far as to give 99 percent of all I get on my conquest here to every constituent of Zinodwo.''

``And that's why they're beggars.\footnote{Pacoitz is bad at formatting dialog, to clarify, this paragraph is said by the beggar.} Because they don't have to work for the money they get, a great lot of them take all the money they get and spend it on casinos, women, and strong drink. And then they have no money left for food so they resort to begging in the streets. So I say unto you that you must like beggars, you've created an awful lot of them, and you like supporting them too. So give me some gold.''

``That's completely ridiculous! All of that is unfounded.'' Said Twizwa.

``I'm not moving off the road till you give me something.'' Said the beggar.

Not being able to put up with it, we gave the beggar a nugget of gold, however, he still did not move.

``Hah! I knew you were a beggar lover.'' He said.

``Come on, we gave you some gold, now move!'' I yelled at the beggar.

``Can't move, I'm a cripple.''

I dragged the beggar off the road and tossed him into a thorn bush. He did not know how to act before a king, and he had offended me by accusing me of making both Zinodwo and Kaaji countries full of beggars. I knew he was wrong -- I really hoped he was.

We went into the city and pitched our tents, and slept.
