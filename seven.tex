\chapter{Doubt}

The next day, we had another meeting with Twizwa and Paavo.

``There we were all ready to destroy that rebel ghost, and he never even showed up and he went behind our backs and robbed our treasury tent.'' I said to Paavo.

``Quite disappointing.'' Said Twizwa, ``But now that we've built the plunder room, no one will be able to steal our spoils.''

``Not unless they can find the key.'' I said as I pulled them out of my pocket and set them on the table, ``The ghost won't be able to take them from me, I've beaten him before. We'll take three quarters of the men with us, that way if it does attack we'll overwhelm it by sheer numbers. And notice how he only attacks us in the woods? If we were to take the plains he would not attack, and even if he were to, we could see him coming and shoot him before he got there. Also instead of using carts (we really only need them for delivery day when our horses have too much to carry) we could select six men to carry the spoils on their horses, that way they could easily outrun this ghost. Believe me, this plan is foolproof.''

``Why, that's genius.'' Said Twizwa.

``Yeah, that's a good plan.'' Said Paavo, ``I hope you bring him down.''

``Well, that's enough for the meeting,'' We said, ``Twizwa, make sure all these things are prepared.''

Twizwa walked out of my tent, and I followed him. I knew that I had left the key on my desk; because I wanted Paavo to take it. I didn't want to see Paavo try to face my army and myself and lose. No, instead he'd take the easy route and open the plunder room and take what he could carry. It was no longer raining, so I knew that my men would stop him before he took all the treasure, so there wouldn't be much harm done to our income.

Paavo had walked into a flock of wolves in sheep clothing, where I was the only real sheep.\footnote{This terrible attempt at a metaphor is not in the original.} 

So we went on the raid with axes in hand and fire arrows (not lit of course) in our quivers.\footnote{Why Pacoitz would think it necessary to say the arrows in their quivers weren't on fire is beyond me.} We took ways through the plains and stayed away from the woods. We robbed a good many houses and stole some horses.

Everyone was dreading the possibility of encountering the ghost, (Twizwa was making his usual speeches about it) but not I; Paavo had sat in on our meeting, it would be very foolish for him to try to attack us when we had left the keys to the plunder room on my desk in my tent. No, he would use the key to steal some plunder out of the locked room.

We were making our way back to camp when one of my men saw the wooden armor approach -- how stupid Paavo was! My men began to shoot their flaming arrows at him. But he was dodging fairly well. But no one can dodge dozens of arrows at once forever, the arrows eventually hit him and he began to catch fire. Despite that, he was now close enough to fight and be a threat; so the soldiers entrusted with the spoils gallopped off on their horses.

The wooden rebel was now writhing in pain. Before my soldiers could use their axes to break his armor off, the figure cast off his wooden armor and we beheld a man clothed in a billowy yellow cloak.

``He has no form.'' Said Twizwa, ``He's just a ghost now, he can't hinder us any more. Without the armor he has no body and --''

Twizwa stopped speaking because the yellow figure had jumped into the air and knocked him off his horse.

``Twizwa, you fool!'' I cried, ``Breaking his armor didn't hinder it, we freed it!''

The rebel ghost rode Twizwa's horse toward the fleeing mounted plunder carriers;\footnote{That sentence is so awkward.} so we all mounted our horses and went after him. He was gaining on the plunder carriers, but he did not know how to properly ride a horse, and in no time at all, his horse was shot out from under him; however, that did not faze him at all and he was running faster on his feet than the horse had been going.

He out-paced us and leaped onto the back of one of the plunder carriers' horses, kicking its rider off at the same time as mounting it. Then grabbing the bag of plunder, he jumped off the horse and ran towards the woods. We pursued him on our horses but we could not match his speed. One of my men, an expert marksman, almost shot the rebel, but before the arrow hit him, I hurled a knife into the ground ahead of the rebel's foot so that he tripped and the arrow missed him.

``Aww! I missed him!'' I yelled trying not to make it obvious that I had intended to kill him with the knife and not just trip him with it. He rolled and quickly got up, and before we could get to him, he had dissapeared into the woods.\footnote{Pacoitz doesn't make it clear, but the rebel only got away with 1 sixth of that day's plunder; remember, there were 6 mounted men each carrying a portion of it.} 

How amazing his physical ability! But also how foolish his actions! It would have been much easier to just break into the plunder room, I practically handed him the key for goodness sakes! This complaint in my mind was dwarfed by my amazement at his raw speed, how can anyone ever run that fast? Was he a ghost after all? He would still be dead if not for my intervention. Why is rebelling against me worth so much to him that he would nearly die with each encounter? What does he do with this plunder? Does he feed the beggars? How noble his intentions must be, I wish I could be so innocent.

We rode our horses very quickly back to camp. The camp had been broken into and the plunder room was empty. The soldiers who had been guarding it recounted how they had tried to stop a yellow clothed man from breaking into the plunder room, but he was quick and dodged all their arrows and caught all their swords in his hands and tossed them out of reach. He took many trips, but he emptied out the whole plunder room down to the last coin.\footnote{This part did not happen in the original. The rebel did not break into the plunder room at this time, he didn't even try. There was no way he could have done all that in so little time even if he could out-run a horse. Pacoitz was just embellishing the story making it ridiculous.}

``The month is ending.''\footnote{The original actually kept good track of time and had each day labeled. Pacoitz just lazily slops together a bunch of ``the next day,'' ``the next week,'' ``one day,'' et cetera, but worst of all is when he leaves whole days out, completely skipping them from the narrative.} Said Twizwa, ``Do you really think the beggar will be able to calm down a people who are expecting riches but instead get nothing?''

``I don't know.'' We said.

\tbreak

I went to the lumberyard and found Paavo splitting wood.

``So are you loyal to the crown.'' I asked.

``What? No.'' Said Paavo. ``Why would I be loyal to a piece of jewelry?''\footnote{Surpisingly, this is not a Pacoitz pun. This was in the original. This highlights Paavo's innocence and is funny at the same time; Pacoitz's puns just disrupt the story.}

``No, I mean, do you revere and respect we the king?''

``Yes. Yes I do.'' Said Paavo nervously. ``But I do disagree with some things my lord does.''

``I appreciate your honesty, Paavo. What exactly do you disagree with.''

``This whole raid Kaaji and give the spoils to Zinodwo thing seems kinda foolish to me. I've been all over this country, and I can see how impoverished and suffering the people are. I can't help but pity them.''

``You're so naive, Paavo.'' I responded. ``A king can't just go taking care of other people's peoples, I'm responsible for, and must look out for, my own.''

``There are some who think you're not doing so well at that.''

``What! Who?! That crippled beggar? Because if it was him-- was it?!''

``Yes.'' Said Paavo meekly.

I left. I was deeply offended. I had once admired Paavo, but I knew now for certain that he was an enemy of my country. Before I had excused it because he seemed so good intentioned, but good intentions don't matter if you have no knowledge of politics and are hurting by trying to help. Before he had done very little harm, but now he had stolen every last coin!\footnote{In the original, the king was upset because Paavo had stole one of his crowns.} I knew what I was doing and knew how to save a country, who was Paavo or that beggar to think that they knew better? I am the king! Even still, I didn't want to be the one responsible for Paavo's death; so I left and went to Twizwa.

``Twizwa. How did we get rid of this wooden rebel last time?'' We asked.

``Last time he was no problem. We killed him in his armor. But now he's a ghost, and now he's been freed from his armor. We might not be able to kill him, but perhaps if we were to trap him in a trap that immobilizes him, that'd be just as well. But once we move to the next town we'll have to face yet another ghost.''

``Do it then!'' We said. And I left and went for a walk alone (I had not done this before). For the first time I noticed the people, I saw the sad expressions on their faces, the pain in their lives; I had done this; I had ruined them. But they were not my people. Surely their pain was necessary to make Zinodwo's people that much happier.

No matter how much I tried to deny it, that beggar liar had put doubt in my heart, I had to go to Zinodwo and make sure that the people there were not beggars.


