\chapter {Marred and Pierced}

``King, the beggars have showed up in great numbers. They're stirring up a riot.'' Said Twizwa, waking me up.\footnote{There's a gap in the narrative here, four days have actually passed since they set camp in the city.}

``Drat,'' We said, ``you give a beggar some money, and now the whole world knows where to go for hand-outs. Tell them we're not giving them anything!''

``I did that already. But they are insisting.''

``Drive them away!''

``Sir, there's at least 3,000 of them.''

 ``3,000, that is a mess. Well, we have 416 men, they could probably take out no more than six men each -- We're dreadfully outnumbered. Go fetch that beggar from yesterday. I'll bet he's in this somehow.''

They brought the man before us.

``Why is that crowd here?'' We asked him.

``Alms, what else?'' He replied.

``There must be 3,000 people out there.'' I said, ``You sent them didn't you? Don't bother lying, I know you did. What's it going to take to make them leave? And please don't say `alms.'" We said.

``You're cursed.'' Said the beggar, ``Last night\footnote{``Two nights ago'' in the original.} I beheld a man wearing wooden armor, marred and full of arrows. He went from town to town telling all the beggars to come here and recieve treasure from you, the enemy king.''

``That's ridiculous.'' Said Twizwa, ``How can a cripple see what happens towns away when he cannot walk? You're a liar in everything you say.''

``Well, that may be true.'' Said the man, ``But I only stretch the truth when it falls short of reality, in order that it might be more so.''

``More what?'' We asked.

``More itself, more truthfull.'' Replied the beggar.

``That doesn't make any sense.'' Said Twizwa, ``The truth doesn't ever fall short of reality, because reality is truth. And when you stretch the truth you are making it into a lie.'

``But when I stretch it, there's more of it and thus more truth.\footnote{This whole banter about truth only applies to English, and so you can guess that it is another pathetic addition by Pacoitz.}'' Said the man.

``We're done talking with you.'' We said.

``In the last town you were in, a rebel, clad in copper, opposed you and made your raids difficult--''

``Who told you of this?'' I interrupted.

``Twizwa told me.'' The man said.

``I did no such thing.'' Twizwa said.

The beggar began to laugh obnoxiously, ``Since Twizwa first opened his mouth, I saw that he was a master of words, that can hide messages in speech. Apparently, by pure accident, he puts secret messages into everything he utters.\footnote{Pacoitz omits the beggar's backstory about how the beggar was lazy and spent a lot of his time playing with words and could hear the rotations of a person's speech.} He's so good he doesn't even realize what secrets he's telling (which most people won't catch) in his everyday conversation.''

Twizwa blushed and said, ``King! He could have heard about that copper rebel from \emph{any} of those beggars.''

This beggar was a liar. But was there any truth in anything he said?

``The spirits of those rebels you killed are rising from the grave to stop you; their blood cries out from the ground.'' Said the beggar, ``First the copper, now the wooden. If you leave to another town, another warrior slain in battle will meet you there. You couldn't kill the first; this one too shall haunt your every step. You want to know why those beggars are there, because the ghost of this rebel invited them.''
 
``I don't care why they're here.'' We said, ``Can you get rid of them?''

\tbreak

``Pardon me,'' Said Twizwa, interrupting my story, ``but isn't it foolish to think that chest wounds are only fatal when they hit the heart?''

``Twizwa, why didn't you bring that up when I was telling that part?'' I responded, shifting the chains to make myself more comfortable.

``My mind wanders, I only thought of it now.''

I was still on the platform telling my story,\footnote{Here Pacoitz returns back to the frame story he contrived. If this is hard to understand it's because Pacoitz's ability to weave a narrative is very weak. If you remember back in the first chapter, the king is telling this story and this question from Twizwa is a disruption to his telling of it. It makes no sense to have this interruption to Mwefu's story be here instead of when the event actually happens.} he had a very good argument, Paavo should have been dead even if I didn't get his heart, how did he live? I did not answer but instead kept telling my story.

\tbreak

``Bring me to the crowd.'' Commanded the beggar. For that moment he spoke with such authority that I wondered if I indeed was the king or whether this man's greatness surpassed my own;\footnote{This nonsense about the beggar's authority was not in the original; Pacoitz is preparing the story for an awful perversion.} however, this all faded when we brought him to the crowd.
He issued them commands such as, ``Go home,'' ``The king has nothing left, but he invites you to enter his country,'' and ``you'll find no hand-outs here,'' but it was all in vain. The people nearest him heard and they relayed it back to all the others, but not all understood. The ones who heard the beggar tried to leave, but the crowd was so thick they could not. It was evident that what the first part of the crowd was hearing was not the same thing the end of the crowd had heard by repeat.

``This is not working.'' I said to the beggar as I grasped the hilt of my sword ready to slay him.

``Hold on. Give me some time.'' Replied the man and he pointed at Twizwa, ``You, wordy mouth! You know how words change meaning when they are said wrong, say the things that I said, but say them so that the people at the end will hear it without its meaning being lost.''

Twizwa then, using his skill with words,\footnote{Scholars think that Twizwa spoke with a lot of redundancy and that when the meaning of what he said changed by being repeated wrong, the meaning still was what he had intended. The downfall of the Fo\-bwa language is evident in that it is so easy to mishear what someone says due to its monosyllabic nature and lack of formalized sentence structure. While these downsides enable rotational shift poetry, they are downsides for ordinary conversation. Because of this, some people question whether Fobwa is a natural language at all. See Appendix B.} said those things to the crowd, and sure enough the beggars dispersed and were gone.

``I may have got rid of the beggars.'' Said the old man, ``But the curse will still stop you in the end.''

``You didn't get rid of them, Twizwa did.'' We said, ``And also, that curse is nonsense.'' 
Before the annoying beggar could get the last word in, I had two of my men carry him back to the gate of that ruined city.

Since all the beggars were disposed of, we set out on our raid. No curse could stop us. If there was a warrior clothed in abused wooden armor, which I doubted, he would be no problem. We gathered gold and silver and precious stones, but they were not abundant, it took the ransacking of many many homes to find any. It was now getting late and storm clouds had formed over head. In fear of the weather, we headed back towards our camp. But the rain came down in torrents. Every one of us got hold of an umbrella and looked up at the sky, I'm not sure about the science behind it, but the sky was glowing a bright yellow, and the rain was so thick it was like a fog had covered the land.

Though it was night, it wasn't very dark because the sky lit the land, but that didn't do much good since no one could see more than thirty feet through the rain, and when it came to hearing, nothing could be heard but the relentless downpour. At least nothing could be heard till a horse neighed and the sound of clinking gold caught my ear. There was a shadow of a man taking gold out of our cart and placing it into his bag. It was as the beggar had said, his armor was mere wood and it was beaten up and covered in arrows. My men started to fire their bows at him, but it is very hard to shoot when one is holding an umbrella so almost all the arrows missed. But I could see that his ability to evade our shots was hardly sufficient to withstand even this. He tried to dodge, but he did so very poorly. So after a few close calls, he ran away and faded into the rain.

But that was not the last of him, he returned (this time with an empty bag) and began to fill up another bag of our gold. I could not stand for this, I dropped my umbrella (how stupid carrying one of these makes one act. My soldiers could not stop him without the danger of ``getting wet,'' what children!) and came at him with my blade to crack his armor open. But he caught my sword with his gloves which seemed to be made out of teal copper. This wasn't a curse out to get me, this was Paavo. I could have twisted my sword out of his hands and slain him, but how could I? Paavo was not really taking much of our spoils, and I was actually quite impressed by his skill thus far. This wretch would die if not for my intervention; so I twisted the sword out of his hand and made it look like I had tripped so that he could get away.

I wished I had not dropped my umbrella.
