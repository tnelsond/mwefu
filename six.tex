\chapter{New Managment}

The next day it was still raining and I went west through that town of Kwaa\-hwaa\footnote{Dead plant.} to see if we could find Paavo. I found him working at the lumber yard splitting wood.

``Paavo,'' I said, pretending to be surprised, ``what are you doing here?''

He stammered for a moment and said, ``Oh, just splitting wood. You bought all my fruit in Vwi so I came here to Kwaa\-hwaa and found employment splitting wood.''

``Why did you not visit me?'' I asked.

``I could not make it through the crowd yesterday!'' He yelled angrily\footnote{Pacoitz has grossly misinterpreted this sentence. Most scholars agree that Paavo was merely nervous when he said this and was not angry whatsoever.}.

I invited him to our camp and I continued to teach him how to fight (or how to dodge rather).
We started out by throwing small pebbles at him to see how he could avoid them. He flailed about and jumped around like a madman, but still I hit him almost every time (It would have been every time, but really small pebbles are hard to throw both fast and accurately).

``What are you doing?'' We asked him?

``I'm being a moving target.'' He replied.

``Dodging is not about being a moving target.'' I said, ``Think for a moment, a good marksman will aim not for where you are, but where you're going to be. And a bad marksman will aim for where you are and probably hit you by accident when you dodge anyway; it's just as likely that you'll run into a shot by moving as by not moving. But by standing still you have a much better opportunity to dodge because you can start moving in any direction without having to stop first. It's best to stay still until you know that you'll be hit unless you don't move.
Dodging is not about moving quickly, it's about only moving when you have to.''

After a speech like that, you would have thought that his evasion abilities would have instantly improved, but changing one's instincts is not easy. What I said was counterintuitive, it would take a while to learn.

\tbreak

I invited Paavo to observe a meeting between Twizwa and I; and he came. Twizwa was discussing ways of defeating these ghosts.

``Well, we've seen from our encounter with the copper rebel that these ghosts can not be killed. My king, you stabbed one through the heart! I think the ghost must have possessed the armor itself.'' Said Twizwa.

``So you're saying that if we destroy the armor then the ghost will be useless against us?'' I asked.

``Yes.'' Said Twizwa, ``I think it would be wise to have each of your soldiers carry an ax rather than a sword.''

``Wouldn't fire arrows be useful too against a wooden menace?'' Said Paavo as his eyes darted around, refusing to meet mine.

Paavo was apparently trying to ward off suspicion. I knew fire arrows would be brought up, hence why I taught him how to dodge, but I did not expect Paavo to be the one doing it.

``Excellent suggestion, Paavo.'' I replied, ``Alright, let's arm our men with fire arrows and axes.''

``Did you ever find out if my family was still alive?'' Asked Paavo.

``My messengers found none of your family, just some old friends of your parents. They didn't really have much to say about you.''

``Oh\ldots So I guess I've no real reason to go back then.'' He said.

So Paavo went back to splitting wood (That's what he said anyway), and my soldiers and I went out on a raid.
We stole some more precious metals and stones, as well as wine, and we took the path through the woods back to camp.

``Odd,'' I thought to ourself, ``I expected Paavo to try to steal some of our plunder.''

No sooner had I finished thinking this when a voice called out, ``King, is that you?!''

``Yes, it is we.'' I replied.

One of the soldiers who we had been guarding the camp ran up to me.

``My lord,'' He said, ``While you were out we were attacked and robbed.''

``How can this be?'' Said Twizwa, ``You had three quarters\footnote{Actually 30} of your men guarding the camp.''

``It rained so hard we could hardly see, and hearing was also difficult.'' The soldier said. It was still raining heavily and ironically\footnote{More of Pacoitz ruining the narrative by trying to be funny.} I didn't catch all that he said and so I asked him to repeat it.

``We were all spread out keeping watch, and the ghost came upon us one by one and tied us up. He then broke into your treasury and took a great deal of it. I managed to untie myself and the others and have run all this way to tell you. I know not whether he is still defiling your plunder.''

So we hurried back to camp and found a great deal of the tents torn through and the treasury one especially. Half of our plunder was gone.

``We've all been fools.'' Said Twizwa, ``Keeping treasure in a tent is the least safe place. If it pleases the king, let us build a stone room with a heavy iron door within which to store our spoils.''

``Let it be done.'' We said.

It took only a day, but it was a very strong building; we moved all our spoils into it.

\tbreak

We continued to go on raids for the next week.\footnote{Pacoitz left out a great scene where king Mwefu talks to Twizwa about the economics of their country and how to keep it going. Twizwa also sends out a report of how much treasure they've acquired to Zinodwo.} The whole time my men were obedient in carrying their firearrows and axes; some were dissapointed that they didn't get to use them and had to carry them anyway, but most were glad they didn't have to use them.

The plunder room started to fill up and did not get broken into. However, Vwumwaa, the man that distributed the plunder to my people in Zinodwo, sent another whiney letter about needing more gold to calm the people down so that they wouldn't revolt.

``What are the people complaining about?'' Said Twizwa, ``The only reason they're getting any share in the treasure is because you are so generous.''

``Vwumwaa always complains even when there's plenty.'' We said, ''And we sent the report last week after we were robbed, so we have much more than he thinks. I tell you, he's more annoying than that beggar\ldots''

Twizwa laughed, ``Then why not put the beggar in charge?''

``Actually, that's not a bad idea--'' I said.

``Sir, you can't be serious.'' Said he.

``Hear me out.'' I said, ``That beggar has been a bother to me, sending him away would solve that problem. He's proven to have a way talking with beggars and stopping mobs, even if he did need your help with such a large crowd. Now I still don't think that the people of our nation are beggars, but perhaps he can deal with them just the same. But how embarassing for Vwumwaa, he will hear that he's been replaced by a beggar of the enemy's country and perhaps he will come groveling back to us and stop whining and annoying me so much.''

``Have you gone mad? You can't just put a beggar, let alone a beggar of the opposing country in charge of an important office such as that!'' said Twizwa.

``Have you forgotten what you were before I promoted you to second in command?'' I replied, ``You were a page, if you value your position, don't question my judgement.''

The beggar was then put in charge of spoil distribution.\footnote{If you can't tell by the complete utter absurdness of all this, Pacoitz made it all up. In the original, the king did entertain the idea of making the beggar responsible for managing the plunder; but he never actually did it, and he surely didn't tell Twizwa about it. Apparently Pacoitz became so enamored with the idea that he just had to ruin the whole credibility of the narrative.
Pacoitz needs to realize that he is supposed to be the translator, not the author!}

