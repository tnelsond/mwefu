\chapter[Prologue]{Prologue
\footnote{This is Robert Graft, unfortunately, if you read the introduction you would know that I would have to pop up in the footnotes to correct things. I told you not to read this book; but since you are, the first chapter is not really a prologue by any stretch of the imagination; actually this chapter comes from the middle of the original book, but Pacoitz never was very bright and wanted to rearrange everything to make it a frame story. And the first paragraph is not even in the original text, it just comes off as Pacoitz trying to be philosophical.}
}

Others always rose up to oppose me;
I wouldn't mind so much if it was just them.
For when I count my enemies and find myself in their ranks, I realize what a wretched creature I am.
It's possible (though highly improbable) that I could overcome everyone else, but how could I possibly overcome myself since I am my own equal?

\tbreak

Finally they had done it;
the soldiers captured the armor clad rebel who had for so long defied their illegitimate and greedy rule.
The nation called Zinodwo had been raiding the poor country (The country wasn't so poor until after it got raided) of Kaaji for almost a decade.
Many heroes had risen up to stop the soldiers of Zinodwo, but this one was the best -- this one also just happened to be me.

They led me up the scaffold, I could hardly walk since they had me bound with so many nets and chains. I knew what was going to happen, they began to tie the noose; I was to be killed like all the heroes before me. But then a soldier removed my helmet.

``My Lord!? He's our King!'' Shouted the soldier in bewilderment.
For I was.
And now it was clear that the very same person that was giving the orders was also the one thwarting them.
But that didn't make any sense, what would I the king have to gain by losing the war? Surely not money because however much Kaaji was paying me I could get more by simply taking it from them.

\indent ``Explain yourself.'' Said the king's second in command.
(Pardon my confusing habit of switching point-of-views, as a king this tends to happen: like using \emph{we} in place of \emph{I})
\footnote{Note that in Fo\-bwa, kings do not talk about themselves by using \emph{we} or anything similar, but Pacoitz doesn't care about historical accuracy as long as he can get a laugh.}
I was very much ashamed of what I had done.
The crowd consisted mostly of Kaajin
\footnote{This means people from the country Kaaji which scholars tell me is Fo\-bwa for ``weak (soft) place''.} with some of my Zinodwan
\footnote{People from Zinodwo, Fo\-bwa for ``rare place''.} soldiers to control them. Despite this, all the people -- in unison -- began to chant, ``Explain!'' and kept on screaming until, ``If you'll be quiet I might just let myself explain!'' I shouted. 

I knew that both sides wanted to kill me; so I began to tell my story.
